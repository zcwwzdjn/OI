\documentclass{noithesis}

\theoremstyle{plain}      \newtheorem{theorem}{定理}[subsection]
\theoremstyle{definition} \newtheorem{problem}{问题}[subsection]

\renewcommand{\emptyset}{\varnothing}

\begin{document}

%% 论文开始

\title{浅谈容斥原理}
\author{成都七中~~王迪}

\maketitle

\begin{abstract}
本文从计数问题中的容斥原理出发,得出容斥原理的形式,通过一些例题探究了
使用它解题的思维方式,然后研究了容斥原理的推广,对容斥原理的一般化作出尝试,
最后总结了容斥原理及其运用过程中所体现的思想、方法。
\end{abstract}

\section{引言}

在一类组合计数的问题中,我们需要对一些集合的并或交中元素的个数进行统计,而
对于这种问题,容斥原理是一种通用的解法,所以在本文的前半部分,我们将探究
容斥原理的形式,用容斥原理解决计数问题的分析方法,以及若干有趣的例题。\par

而容斥原理不仅可以解决组合计数问题,在本文的后半部分,我们将对容斥原理进行推广,可以
解决一些数论和概率论中的问题,而通过分析不同问题中容斥原理的形式,我们可以将容斥原理
一般化,从更高的层面理解容斥原理。 \par

\section{容斥原理}

在这一小节中,我们会从一些组合计数问题出发,得出容斥原理的形式并给出证明,
然后通过一些例题得出用容斥原理解题的思维方法。

\subsection{预备知识}

考虑一个简单的问题:某班有$a$个人擅长唱歌,$b$个人擅长画画,$c$个人既擅长唱歌
也擅长画画,问多少人有至少一种特长? \par

通过画文氏图的方法,很容易得出此问题的答案:$a+b-c$。 \par

我们可以得出该问题的一般形式:设一个有限集为$U$,$U$中元素有两种性质$P_1$和$P_2$,而满足$P_1$性质
的元素组成集合$S_1$,满足$P_2$性质的元素组成集合$S_2$,那么上面的问题相当于是求至少满足两种性质之一
的元素个数,可以表示成这样:
\[
|S_1 \cup S_2| = |S_1| + |S_2| - |S_1 \cap S_2|
\]
其中$|S|$表示集合$S$中的元素个数。 \par

我们考虑$U$中元素有三种性质$P_1,P_2,P_3$,对应的子集是$S_1,S_2,S_3$,仍然可以通过画
文氏图的方法,得到下面的等式:
\[
|S_1 \cup S_2 \cup S_3| = |S_1| + |S_2| + |S_3| - |S_1 \cap S_2| - |S_1 \cap S_3| - |S_2 \cap S_3| + |S_1 \cap S_2 \cap S_3|
\]
注意到,我们把\textbf{求并集中元素个数}转化成了\textbf{求交集中元素个数},这一步体现了\textbf{转化}的思想 。\par

一般地,设$U$中元素有$n$种不同的性质,第$i$种性质称为$P_i$,满足$P_i$的元素组成集合$S_i$,那么
\begin{theorem}
满足$P_1,P_2,\cdots,P_n$中至少一个性质的$U$中元素的个数是
\begin{eqnarray*}
|S_1 \cup S_2 \cup \cdots \cup S_n| & = & \sum_{i}{|S_i|} - \sum_{i < j}{|S_i \cap S_j|} + \sum_{i < j < k}{|S_i \cap S_j \cap S_k|} - \cdots \\
                                    &   & \cdots + (-1)^{n - 1}{|S_1 \cap S_2 \cap \cdots \cap S_n|}
\end{eqnarray*}
\end{theorem}
\begin{proof}[证明]
考虑一个处于$\bigcup_{i=1}^{n}{S_i}$中的元素$x$,它所属$m$个集合$T_1,T_2,\cdots,T_m$,那么我们计算一下上式右边
统计的$x$个数$cnt$:
\begin{eqnarray*}
cnt & = & |\{T_i\}| - |\{T_i \cap T_j| i < j \}| + \cdots + (-1)^{m - 1}|\{T_1 \cap T_2 \cap \cdots \cap T_m\}| \\
    & = & C_m^1 - C_m^2 + \cdots + (-1)^{m-1}C_m^m \\
    & = & -\left(\biggl(\sum_{i=0}^m{(-1)^i C_m^i}\biggr) - C_m^0\right) \\
	& = & -\biggl((1 - 1)^m - 1\biggr) \\
	& = & 1
\end{eqnarray*}
结合二项式定理即可证明容斥原理的正确性。
\end{proof}
\par

至此,我们已经知道了用容斥原理计算集合的并中元素的数目的方法。稍加变形我们就能用容斥原理计算集合
的交中元素的数目。 \par

我们用$\bigcap_{i=1}^{n}{S_i}$表示需要计数的交集,令$\overline{S}$表示集合$S$关于
全集$U$的补集,那么$\overline{S_i}$就表示不满足性质$P_i$的集合。考虑到需要计数的
是满足$P_1,P_2,\cdots,P_n$的元素,我们进行一步\textbf{补集转化},求不满足至少一个性质的
$U$中的元素个数,即$\bigcup_{i=1}^{n}{\overline{S_i}}$,这个集合的计数方式就和之前
类似,所以
\begin{theorem}
满足$P_1,P_2,\cdots,P_n$中所有性质的$U$中元素的个数是
\[
|S_1 \cap S_2 \cap \cdots \cap S_n| = |U| - |\overline{S_1} \cup \overline{S_2} \cup \cdots \cup \overline{S_n}|
\]
\end{theorem}
\par

综上所述,我们可以利用容斥原理在\textbf{求并集元素个数}和\textbf{求交集元素个数}这两个问题间\textbf{互相转化},
这提示我们,用容斥原理解题,是一个\textbf{转换角度}的思维方式。

\subsection{经典问题}

我们通过几个组合计数的经典题目,来探究如何应用容斥原理。

\subsubsection{不定方程非负整数解计数}

\begin{problem}
考虑不定方程
\[
x_1+x_2+\cdots+x_n=m
\]
和$n$个限制条件$x_i \le b_i$,其中$m$和$b_i$都是非负整数,求该方程的非负整数解的数目。
\end{problem}
\par

在解决这个问题之前,这里不加证明地给出一个结论:
\begin{theorem} \label{theorem:eqsol}
不定方程$x_1+x_2+\cdots+x_n=m$的非负整数解数目为$C_{m+n-1}^{n-1}$。
\end{theorem}
\par

在应用容斥原理前,我们需要找出全集$U$,以及刻画$U$中元素的$P_i$。
\begin{itemize}
\item $U$是满足$x_1+x_2+\cdots+x_n=m$的所有非负整数解;
\item 对于每个变量$i$,都对应一个$P_i$,而$P_i$代表的性质是$x_i \le b_i$。
\end{itemize}
设满足$P_i$的所有解组成集合$S_i$,那么我们需要求解的值就是:$|\bigcap_{i=1}^n{S_i}|$。 \par

由之前的知识我们可以写出:$|\bigcap_{i=1}^n{S_i}|=|U|-|\bigcup_{i=1}^n{\overline{S_i}}|$。而$|U|$的值
可以由定理\ref{theorem:eqsol}计算,我们着重考虑后面的部分,而这正是之前容斥原理的一般形式! \par

通过展开$\bigcup_{i=1}^n{\overline{S_i}}$,问题转化成:对于某几个特定的$\overline{S_i}$,求解满足这些条件
的解的数目。一般地,给出$1 \le i_1 < i_2 < \cdots < i_t \le n$,求$|\bigcap_{k=1}^t{\overline{S_{i_k}}}|$。 \par

考虑$\overline{S_{i_k}}$的含义,即满足$x_{i_k} \ge b_{i_k} + 1$的解的数目。而对每一个$k$,都要满足
这个条件,即\textbf{部分变量有下界限制},我们可以在方程的右边减去下界和$\sum_{i=1}^k{(b_{i_k}+1)}$,那么
新方程的解与我们要求的解是一一对应的!而新方程的每个变量都没有上下界限制,所以同样可以用定理\ref{theorem:eqsol}
求出。 \par

于是我们只需要枚举$\{\overline{S_1},\overline{S_2},\cdots,\overline{S_n}\}$的非空子集,进行容斥原理的计算即可。 \par

考虑解题过程,我们先是把问题写成集合的形式,找出全集$U$,以及我们的解需要满足的性质$P_i$,然后写出需要求值的式子,
用容斥原理进行展开,于是我们可以着眼局部,这时的限制数就大大减少,成为一个个可解的问题,最后我们把答案合并起来就可以了。

\subsubsection{错位排列计数}

\begin{problem}
称一个长度为$n$的排列$p$为错位排列,当且仅当对所有的$1 \le i \le n$,都满足$p_i \neq i$。给出$n$,求长度为$n$
的错位排列的数目。注意排列中$1$到$n$的整数都恰好出现$1$次。
\end{problem}
\par

同上题,我们首先分析全集$U$和性质$P_i$:
\begin{itemize}
\item $U$表示长度为$n$的所有排列;
\item 对于每个位置$i$,都对应一个$P_i$,而$P_i$代表的性质是$p_i \neq i$。
\end{itemize}
同样设满足$P_i$的解组成集合$S_i$,那么我们需要求的值仍是$|\bigcap_{i=1}^n{S_i}|$! \par

于是用同样的处理方法,我们写出$|\bigcap_{k=1}^t{\overline{S_{i_k}}}|$,我们考虑$\overline{S_{i_k}}$的含义,
即$p_i = i$的排列数目,而对每一个$k$,都确定了排列中一个位置的数,所以共有$t$个位置的数被确定了,
而其他位置是没有限制的,所以对应的答案就是$(n-t)!$。 \par

进一步可以推出,只要我们枚举的$\{\overline{S_i}\}$的子集的大小一样,它们对答案的贡献也是一样的!设长度为$n$的
错位排列数是$D_n$,那么我们有:
\begin{eqnarray*}
D_n & = & n! - \sum_{t=1}^n{(-1)^{t-1} \sum_{i_1 < i_2 < \cdots < i_t}(n-t)!} \\
    & = & n! + \sum_{t=1}^n{(-1)^t C_n^t (n-t)!} \\
	& = & n! + \sum_{t=1}^n{(-1)^t \frac{n!}{t!}} \\
	& = & n! \sum_{t=0}^n \frac{(-1)^t}{t!}
\end{eqnarray*}
\par

由此我们发现,用容斥原理解决问题的时间复杂度不一定是指数级,我们可以对一些对答案贡献一致的情况进行
合并,这样仍能得出高效的算法。 \par

另外,错位排列数$D_n$也有递推的方法,有兴趣的同学可以另行探究。

\subsection{例题解析}

下面通过一些例题,看一看容斥原理在信息学中的应用。

\subsubsection{HAOI2008~硬币购物}

\begin{problem}
有$4$种面值的硬币,第$i$种硬币的面值是$c_i$。有$n$次询问,每个询问中第$i$种硬币的数目是$d_i$,
以及一个购物款$s$,回答付款方法的数目。数据规模$n \le 10^3,s \le 10^5$。
\end{problem}
\par

这题初一看是一个经典的多重背包问题,但是经过分析,我们发现单次动态规划的最好复杂度是$O(4s)$,对于
多次询问根本无法承受。 \par

但是这题与一般的背包问题有一个明显的不同:硬币(即不同的物品)只有$4$种。而且,若每次购物没有硬币
数目的限制,可以用一个动态规划预处理后$O(1)$回答每组询问。 \par

考虑一次询问,第$i$种硬币使用的数目是$x_i$,那么需要满足$c_1x_1+c_2x_2+c_3x_3+c_4x_4=s$,且$x_i \le d_i$。
我们发现,这与之前的不定方程非负整数解计数非常类似,只不过每个变量前有一个系数。 \par

同样我们用容斥原理来处理这个问题,$S_i$表示满足$x_i \le d_i$的解的数目,$\overline{S_i}$表示满足$x_i \ge d_i + 1$
的解的数目,考虑若干$\overline{S_i}$的交集,即一些硬币使用数有下限,我们同样可以从$s$中减去下界和,
问题变成了对于一个$s'$,若硬币使用数目无限制,有多少种不同的付款方式。而这是一个经典的无限背包问题,
可以预处理。 \par

所以对每组询问进行容斥,设最大的$s$为$m$,那么总的时间复杂度就是$O(4m + n \cdot 2^4)$。 \par

考虑我们的解题过程,我们首先发现问题的经典算法时间复杂度过高,但是我们抓住了题目的特殊性,通过写出问题
的数学形式,通过联想,应用容斥原理把问题拆分,减少了局部问题的限制数,最终解决了问题。

\subsubsection{原创题~游戏}

\begin{problem}
Alice和Bob在玩游戏。他们有一个$n$个点的无向完全图,设所有的边组成了集合$E$,于是他们想取遍
$E$的所有非空子集,对某个集合$S$有一个估价$f(S)$,这个估价是这样计算的:考虑$n$个点与$S$中的
边组成的图,我们用$m$种颜色对所有点染色,其中同一个联通块的点必须染成一种颜色,那么$f(S)$等于
这个图的染色方案数。同时,Alice喜欢奇数,所以当$|S|$为奇数时,Alice的分值加上$f(S)$,否则Bob的分值
加上$f(S)$。求最后Alice的分值减去Bob的分值的值模$10^9+7$的结果。数据规模$n,m \le 10^6$。
\end{problem}

显然我们无法枚举$E$的所有非空子集;另一方面,对于相同的$|S|$,联通块数目也不尽相同。我们似乎找不到
一个突破口。这种情况下,我们就应该写出问题的数学形式,再进行分析。 \par

首先,一个事实是,“同一联通块必须染相同的颜色”与“有边直接相连的两点必须染相同的颜色”是等价的。
于是我们可以对每个点设一个变量,用$x_i$表示第$i$个点的颜色,$x_i$是$[1,m]$中的整数,那么
一条无向边$(i,j)$就表示一个等式$x_i = x_j$。我们考虑Alice的得分$scoreA$和Bob的得分$scoreB$,
令$F(C)$表示在情况$C$下的染色数,用$[C]$表示一个情况$C$,则
\begin{eqnarray*}
scoreA & = & \sum_{\emptyset \neq S \subseteq E,|S|\mbox{是奇数}} F\left( \bigcap_{(i, j) \in S} [x_i = x_j] \right) \\
scoreB & = & \sum_{\emptyset \neq S \subseteq E,|S|\mbox{是偶数}} F\left( \bigcap_{(i, j) \in S} [x_i = x_j] \right)
\end{eqnarray*}
现在考虑$ans = scoreA - scoreB$,即$|S|$为奇数时贡献为正,$|S|$为偶数是贡献为负,容易想到加一个$-1$的幂将式子统一:
\begin{eqnarray*}
ans & = & \sum_{\emptyset \neq S \subseteq E} (-1)^{|S|-1} F\left( \bigcap_{(i, j) \in S} [x_i = x_j] \right)
\end{eqnarray*}
我们把$[x_i = x_j]$这$\frac{n(n+1)}{2}$个情况用$P_i$代替,令$t = \frac{n(n+1)}{2}$,则$P_i$的$i$的取值范围是$1 \le i \le t$。
令$Q = {P_i}$,那么再考虑上式:
\begin{eqnarray*}
ans & = & \sum_{\emptyset \neq S \subseteq Q} (-1)^{|S|-1} F\left( \bigcap_{P_i \in S} P_i \right) \\
    & = & \sum_{i}F(P_i) - \sum_{i < j}F(P_i \cap P_j) + \cdots + (-1)^{t - 1} F(P_1 \cap P_2 \cap \cdots \cap P_t)
\end{eqnarray*}
注意到这个形式与容斥原理极其相似!我们可以根据容斥原理,逆向分析出上式右边所求值的含义,即
\begin{eqnarray*}
ans & = & F\left( \bigcup_{i = 1}^t{P_i} \right)
\end{eqnarray*}
考虑上式右边的含义,即\textbf{至少有两个点颜色相同}的染色数!那么该问题中全集是点的染色方案集合,通过补集转化,我们
就只需要求\textbf{点两两颜色不同}的染色数!而这个的计算方法是显然的,答案是$\prod_{i = 1}^n{(m - i + 1)}$。所以
原问题答案就是$m^n - \prod_{i=1}^n(m-i+1)$。 \par

细心的同学应该发现了,上面的式子中存在一个函数$F$,它对一个情况,即一些条件的交定义,其实我们考虑满足$P_i$的染色方案构成
集合$S_i$,那么其实$F\left( \bigcap P_i \right) = | \bigcap S_i |$,这样就和之前的容斥原理形式一致了。 \par

回顾我们的解题过程,我们首先直接写出了答案的数学形式,把一些文字条件转化为数学条件,再进行一些换元、代入,
得到一个关于若干条件的交集的式子,最终得到容斥原理的形式,\textbf{逆向分析}出问题的本质,找出算法并解决问题。
如果说原来的容斥原理都是通过着眼局部,整合答案,在某种意义上进行了“微分”,那么这道题目中我们就是用的整体分析
的方法,对答案的一个冗长的式子进行了“积分”,得到一个简洁的答案。这两个方向都体现了信息学中的\textbf{转化}思想。

\section{容斥原理的推广}

\subsection{数论中的容斥原理}

我们考虑一个经典的问题:给一个正整数$n$,求$1$到$n$中与$n$互质的数的个数$\varphi(n)$。 \par

事实上我们要求的是$|\{ x | 1 \le x \le n, gcd(x, n) = 1 \}|$,其中$gcd(a,b)$表示$a$和$b$的最大公约数。注意到
这也是一个对某个集合计数的问题,但是$gcd(x,n)=1$这个限制太“大”,因为$gcd$这个函数本身较“复杂”,所以,我们应该
想到从最大公约数的性质,去把$gcd(x,n)=1$这个限制拆成若干个小的限制。 \par

我们考虑两个数$a$和$b$的质因数分解,若$a=p_1^{a_1}p_2^{a_2} \cdots p_k^{a_k}$,$b=p_1^{b_1}p_2^{b_2} \cdots p_k^{b_k}$,
那么我们有$gcd(a, b) = p_1^{min(a_1,b_1)}p_2^{min(a_2,b_2)} \cdots p_k^{min(a_k,b_k)}$,其中$min(x,y)$表示$x$
和$y$中的较小值。 \par

注意到,若两个数$a$和$b$的最大公约数是$1$,那么它们的因数分解中一定没有相同的质数,而这是一个充要条件!所以,
若$n$的不同的质因子有$p_1,p_2,\cdots,p_k$共$k$个,那么我们需要统计的$x$就要同时满足$k$个条件,即对于$1 \le i \le k$,
都有$x$不是$p_i$的倍数。 \par

现在我们可以把我们的结论写成数学的形式。设$P_i$表示$x$不是$p_i$的倍数这个性质,$S_i$表示$1$到$n$中满足$P_i$的数组成的集合,
那么这里的全集$U$就是$1$到$n$的整数集合,我们需要求的就是:
\begin{eqnarray*}
|\{x | 1 \le x \le n, gcd(i, n)=1\}| & = & \left| \bigcap_{i=1}^{k} S_i \right| \\
                                     & = & |U| - \left| \bigcup_{i=1}^{k} \overline{S_i} \right| \\
									 & = & n - \left| \bigcup_{i=1}^{k} \overline{S_i} \right|
\end{eqnarray*}
这就是一个容斥原理的式子! \par

再考虑$\cap_i{\overline{S_i}}$的含义,它表示的是对于一些质数,我们统计$[1,n]$上有多少个数同时是这些数的倍数。
这个的统计方法非常简单:设质数的积为$m$,那么答案就是$\frac{n}{m}$。 \par

所以我们可以写出我们所求答案的表达式:
\begin{eqnarray*}
|\{x | 1 \le x \le n, gcd(i, n)=1\}| & = & n - \sum_{i} \frac{n}{p_i} + \sum_{i < j} \frac{n}{p_ip_j}-\cdots +(-1)^k\frac{n}{p_1p_2 \cdots p_k} \\
                                     & = & n\left(1-\frac{1}{p_1}\right)\left(1-\frac{1}{p_2}\right)\cdots \left(1-\frac{1}{p_k}\right) \\
									 & = & n \prod_{i=1}^k \left(1 - \frac{1}{p_i}\right)
\end{eqnarray*}
这其实就是著名的欧拉公式。 \par

从这个例子可以发现,我们容斥时考虑的是一个质数的集合,而我们取遍这个集合的子集时,得到的质数的乘积中所有质因子的次数都是$1$,
我们称这样的数为\textbf{无平方因子数}。再看$1$到$n$中每个$n$的约数对答案的贡献,显然只有$1$和无平方因子数有贡献,而且无平方因子数
所作贡献的正负与质因子的个数有关。定义一个函数$\mu(n)$,它定义在正整数集合上,且
\[
\mu(n) = \left\{
	\begin{array}{ll}
		1 & n = 1 \\
		(-1)^k & n=p_1p_2 \cdots p_k \\
		0 & \mbox{otherwise}
	\end{array} \right.
\]
这其实就是著名的\textbf{莫比乌斯函数}。我们再重新考虑之前的问题,容斥过程中的表达式可以写成$\varphi(n) = \sum_{d|n} \mu(d) \frac{n}{d}$。 \par

数论中的很多计数问题都可以用类似的方法解决:考察“最小元”即质数,计算“部分”即每个约数对答案的贡献,利用莫比乌斯函数进行容斥。在数论中,
还有一种方法叫\textbf{莫比乌斯反演},有兴趣的同学可以另行探究。

\subsection{概率论中的容斥原理}

在概率论中,对于一个概率空间内的$n$的事件$A_1,A_2,\cdots,A_n$,也存在着一个容斥原理:
\[
\mathbb{P}\left(\bigcup_{i=1}^n A_i \right) = \sum_{i=1} \mathbb{P}(A_i) - \sum_{i<j} \mathbb{P}(A_i \cap A_j) + \cdots + (-1)^{n-1} \mathbb{P}(A_1 \cap A_2 \cap \cdots \cap A_n)
\]
若事件的交集发生的概率只和事件的数量有关,且设$k$个事件的交集的概率为$a_k$,那么可以用组合数简化:
\[
\mathbb{P}\left(\bigcup_{i=1}^n A_i \right) = \sum_{k=1}^{n} (-1)^{k-1} C_n^k a_k
\]
\par

容斥原理在概率论中的实际应用比较少见,笔者也没有用容斥原理解决概率问题的经验。这个领域仍需更深一步的探究。

\section{容斥原理的一般化}

\subsection{预备知识}

由前面可知,容斥原理适用于对集合的计数问题,其实,对于两个关于集合的函数$f(S)$和$g(S)$,若
\[
f(S) = \sum_{T \subseteq S} g(T)
\]
那么就有
\[
g(S) = \sum_{T \subseteq S} (-1)^{|S| - |T|} f(T)
\]
\par

这是一个更加一般的形式,而且对于之前讨论过的几种情形下的容斥原理都能找到$f(S)$和$g(S)$函数进行对应,其中
$S$表示的是$n$个性质的集合。由于找到的$f(S)$和$g(S)$形式很复杂,在此略过,有兴趣的同学可以参考维基百科“容斥原理”
词条。 \par

另外,上面的式子也可以稍加变形写成这样:
\begin{eqnarray*}
f(S) & = & \sum_{T \supseteq S} g(T) \\
g(S) & = & \sum_{T \supseteq S} (-1)^{|S| - |T|} f(T)
\end{eqnarray*}
其实只用把之前式子中$S$和$T$替换成关于全集的补集,$\subseteq$号就换成$\supseteq$了。 \par

下面我们通过一个例子来感知一下。

\subsection{例题:有标号DAG计数}

\begin{problem}
给出$n$,对$n$个点的有标号有向无环图进行计数,输出答案模$10^9+7$的值。数据规模$n \le 5 \times 10^3$。
\end{problem}
\par

这是一类图的计数的问题。我们考虑动态规划,因为有向无环图中有一类特殊的点,即$0$入度的点,所以记
$dp(i,j)$表示$i$个点的有向无环图,其中恰有$j$个点的入度为$0$,的答案,那么我们考虑去掉这$j$个点
后,还有$k$个点入度为$0$,写出转移
\[
dp(i,j) = C_i^j \sum_{k=1}^{i-j} (2^j - 1)^k 2^{j(i - j - k)} dp(i - j, k)
\]
$C_i^j$表示从$i$个点中选出$j$个点的选法,而去掉$j$个点后的$k$个$0$入度点与这$j$个点间至少有$1$条边即
$(2^j -1)^k$,然后这$j$个点还可以往除了这$j+k$个点之外的点随意连边即$2^{j(i - j - k)}$。这个算法
时间复杂度$O(n^3)$。 \par

注意我们在定义状态时,是“$0$入度点恰好为$k$”,因为限制过严,导致我们需要考虑的很多。一个常见的办法是,
在状态定义中将“恰好”改成“不少于”以放宽限制。但在这个问题中,从$i$个点选不少于$j$个$0$度数点,选法很多,
转移时重复计算的情况很复杂,我们可以考虑将这不少于$j$个的点特殊化,即 \par

我们记$f(n, S)$表示$n$个点,只有$S$中的点的入度为$0$;类似地定义$g(n, S)$表示$n$个点,至少$S$中的点的
入度为$0$。可以发现$g(n, S)$的转移比较简单:
\begin{eqnarray}
g(n, S) = 2^{|S|(n - |S|)}g(n - |S|, \emptyset) \label{eqnarray:one}
\end{eqnarray}
另一方面,我们再考虑$f(n,S)$和$g(n,S)$的关系,这也比较简单:
\begin{eqnarray}
g(n, S) = \sum_{T \supseteq S} f(n, T) \label{eqnarray:two}
\end{eqnarray}
注意式子\eqref{eqnarray:two}与之前提到的一般化的容斥原理相似,不妨将之应用:
\begin{eqnarray}
f(n, S) = \sum_{T \supseteq S} (-1)^{|S| - |T|} g(n, T) \label{eqnarray:three}
\end{eqnarray}
而我们的目的是求$g(n, \emptyset)$,先使用式子\eqref{eqnarray:two}进行推导:
\begin{eqnarray*}
g(n, \emptyset) & = & \sum_{\emptyset \neq T} f(n, T) \\
                & = & \sum_{m=1}^n \sum_{|T|=m} f(n, T)
\end{eqnarray*}
再代入我们用容斥原理推出的式子\eqref{eqnarray:three}:
\begin{eqnarray*}
g(n, \emptyset) & = & \sum_{m=1}^n \sum_{|T| = m} \sum_{S \supseteq T} (-1)^{|T| - |S|} g(n, S) \\
                & = & \sum_{m=1}^n \sum_{|T| = m} \sum_{S \supseteq T} (-1)^{|T| - |S|} 2^{|S|(n-|S|)} g(n-|S|, \emptyset) \\
				& = & \sum_{m=1}^n \sum_{|T| = m} \sum_{k=m}^n C_{n-m}^{k-m} (-1)^{k-m} 2^{k(n-k)} g(n - k, \emptyset) \\
				& = & \sum_{m=1}^n C_n^m \sum_{k=m}^n C_{n-m}^{k-m} (-1)^{k-m} 2^{k(n-k)} g(n - k, \emptyset)
\end{eqnarray*}
利用一些组合数的性质可以继续进行化简。这里直接给出最后的化简结果:
\[
g(n, \emptyset) = \sum_{k=1}^n (-1)^{k-1} C_n^k 2^{k(n-k)} g(n - k, \emptyset)
\]
注意到此时我们计算$g(n, \emptyset)$的时间复杂度降到了$O(n^2)$,容斥原理在中间起到了举足轻重的作用。 \par

回顾我们的解题过程,首先我们在定义状态时放宽了状态的限制,这样可以认为新的状态是之前状态某种意义下的
“前缀和”,列出等式后用容斥原理得到另一个式子,然后整合我们手中的等式推导答案的表达式,最后得到复杂度较低的
算法。 \par

\section{总结}

容斥原理是组合数学中一个重要的定理,在解决问题的时候,我们既可以使用“隔离法”,将所需求的解要满足的条件拆分,
放宽限制,解决若干简单的子问题,再整合答案;也可以使用“整体法”,对所求的式子进行整体感知,逆向地合并条件,
找出问题的本质。这里体现了\textbf{转化}的思想,当然在思考过程中也需要一些数学功底。 \par

容斥原理同时并不是仅仅应用于组合计数,稍加变形后就可以解决一些数论或概率论的问题,其思想是一致的。而最后我们
通过一些资料得知了容斥原理更为一般的形式,它适用于定义在集合上的函数,这使得容斥原理更加抽象,也让我们开阔了
思路,即在一些情况下,我们用集合的形式描述我们的算法,利用容斥原理得到另外的等式,这相当于增加了已知量,
使得问题更容易入手。 \par

在研究过程中,我从解决计数问题中体会到了一些信息学中的思维方法:转化、特殊化(放宽限制)、逆向分析,开阔了眼界;
同时,在查阅容斥原理相关资料的过程中,意识到了平时学习的各种算法,其背后或许仍有继续研究的空间,所以我们应不断求知,
将学习到的知识有机整合,并思考它们的本质,体会不同的算法后面的思想,形成自己的知识网络,增强自己的思维能力。

\section*{参考文献}

\begin{enumerate}
\item http://en.wikipedia.org/~~维基百科
\item http://www.math.ust.hk/$\sim$mabfchen/Math232/Inclusion-Exclusion.pdf
\item 顾昱洲,《Graphical Enumeration》
\end{enumerate}

\section*{感谢}

\begin{itemize}
\item 感谢父母对我的养育
\item 感谢我的教练成都七中的张君亮老师,以及其他所有给予我支持的老师
\item 感谢罗雨屏、李凌霄、钟皓曦等同学的帮助
\item 感谢CCF给我一个展示自己的机会
\end{itemize}

%% 论文结束

\end{document}
