\documentclass{noithesis}

\begin{document}

首先求一个大小为$n$的无向联通图中,边数为偶数的方案数减去边数为奇数的方案数的值。设边数为偶数的方案数为$f(n)$,边数奇数的方案数为$g(n)$,那么对于$n>1$,可以写出递推式:
\begin{eqnarray*}
f(n) & = & 2^{C_n^2-1} - \sum_{i=1}^{n-2} (f(i) + g(i)) \times 2^{C_{n-i}^2-1} \times C_{n-1}^{i-1} - f(n-1) \times C_{n-1}^{n-2} \\
g(n) & = & 2^{C_n^2-1} - \sum_{i=1}^{n-2} (f(i) + g(i)) \times 2^{C_{n-i}^2-1} \times C_{n-1}^{i-1} - g(n-1) \times C_{n-1}^{n-2}
\end{eqnarray*}
两式相减有
\begin{eqnarray*}
f(n) - g(n) & = & -(n-1)(f(n-1)-g(n-1))
\end{eqnarray*}
又因为$f(1)-g(1)=1$,所以$f(n)-g(n)=(-1)^{n-1}(n-1)!$。\par
然后再来考虑题目所表明的表达式。令$rank(S)$表示$S$形成的联通块个数。
\begin{eqnarray*}
ans = \sum_{S} (-1)^{|S|} m^{rank(S)}
\end{eqnarray*}
考虑枚举$rank(S)$,重写答案表达式
\begin{eqnarray*}
ans = \sum_{k=1}^n m^{k} \sum_{rank(S)=k} (-1)^{|S|}
\end{eqnarray*}
重点在于求$\sum_{rank(S)=k} (-1)^{|S|}$,记为$cnt_k$。我们换一个思路考虑,
若有两个联通块,一个大小为$a$一个大小为$b$,那么两者合起来的$f-g$方案数就是
\[
f(a)f(b)+g(a)g(b)-f(a)g(b)-f(b)g(a)=(f(a)-g(a))(f(b)-g(b))
\]
即,是可以乘起来的!而且这个可以推广到任意多个联通块。于是有
\[
cnt_k = \sum_{s_1+s_2+\cdots+s_k=n} (-1)^{n-k} \prod_{i=1}^k (s_i-1)!
\]
我们发现,这其实是第一类Stirling数,所以
\[
cnt_k = s(n, k)
\]
于是可以代回去化简答案表达式
\begin{eqnarray*}
ans & = & \sum_{k=1}^n m^k \sum_{rank(S)=k} (-1)^{|S|} \\
    & = & \sum_{k=1}^n m^k s(n, k) \\
	& = & (m)_n \\
	& = & \prod_{i=1}^n (m-i+1)
\end{eqnarray*}
\end{document}
