\documentclass{noithesis}

\begin{document}

\title{浅谈容斥原理}
\author{成都七中~~王迪}

\maketitle

各位老师、同学,尊敬的评委们,大家下午好。我是来自成都七中的王迪。
我很荣幸能够站在这里跟大家分享一些我的学习经验,而今天我想要讲的主题是容斥原理。 \par

相信大家在小学时就碰到过这样的数学问题:某班有$a$个人擅长唱歌,$b$个人擅长画画,
$c$个人既擅长唱歌也擅长画画,问有多少人有至少一种特长? \par

我们可以用各种各样的方法来解决这个问题。比如,我们可以利用文氏图,在一张纸上画两个圆,
圆$A$表示擅长唱歌的同学的集合,圆$B$表示表示擅长画画的同学的集合,那么它们的相交区域
$C$就表示既擅长唱歌也擅长画画的同学,由图我们发现,若我们直接将$A$的人数和$B$的人数相加,
会导致$C$中的每个人都被多算一次,所以要减去$C$的人数,即$a+b-c$。 \par

这个问题中我们使用的方法其实就是容斥原理。在论文中,我从
这个最基本的问题出发,通过一些例题讲解了容斥原理在组合计数中的应用,
总结了用解决此类问题的思想方法,然后讨论了容斥原理在其他领域内的
推广,最后研究了容斥原理更加一般的形式。遗憾今天时间有限,我在这里选择
了我的一个原创题与大家分享。当然,我们先来看看容斥原理到底是什么。 \\

类比最开始的那个问题,所有同学组成了一个有限的全集$U$,擅长唱歌称为性质
$P_1$,具有这个性质的同学组成集合$S_1$,擅长画画称为性质$P_2$,具有这个
性质的同学组成集合$S_2$,那么我们可以用数学语言来描述原问题,即,$S_1$并
$S_2$表示至少有一项特长的同学组成的集合,$S_1$交$S_2$表示同时有两项特长
的同学组成的集合,那么我们所求的就是集合$S_1$并$S_2$的大小,根据之前的
答案,我们知道它等于$S_1$的大小加上$S_2$的大小再减去$S_1$交$S_2$的大小,
如屏幕所示。 \par

我们将问题稍加推广,当我们用三种性质描述元素时,仍然用画文氏图的方法
分析,可以得到下面的结论。我们发现,整个表达式是加减交错的,而且将求并集
的问题转化成求交集,于是我们可以归纳出容斥原理的形式。 \par

设我们用$n$种性质$P_1,P_2$到$P_n$描述集合$U$中的元素,满足性质$P_i$的元素
组成了集合$S_i$,那么我们可以得出容斥原理的表达式,如屏幕所示。通俗的说,
就是满足一个性质的减去满足两个性质的加上满足三个性质的,依次类推。这个
式子的证明,我简单提一下,我们可以对每个元素,计算式子两边对它统计的次
数,过程中利用组合数进行化简。

让我们来分析一下这个式子。式子的左边是统计若干集合的并,即至少满足一种性质的
元素个数;而式子的右边被分成了若干个部分,每个部分是在对特定的集合的交集计数,
这提示了我们使用容斥原理解题的一般思路:我们从集合的角度分析问题,找出全集$U$
和若干用来描述元素的性质$P_i$,然后发现问题是求一些集合的并集大小,我们可以
利用容斥原理将其转化,分解为若干的子问题并分别求解,最后合并答案。 \par

关于容斥原理的复杂度,一般来说,因为要枚举子集,对于集合个数$n$是指数级的,
但是,如果我们发现,集合的交集的大小只与集合个数有关,就可以枚举集合的个数,
利用组合数来计算答案了。 \\

下面我们来看一道比较另类的题目:游戏。 \par

Alice和Bob在玩游戏。他们有一个$n$个点的无向完全图,设所有的边组成了集合$E$,于是他们想取遍
$E$的所有非空子集,对某个边集$S$有一个估价$f(S)$,这个估价是这样计算的:考虑$n$个点与$S$中的
边组成的图,我们用$m$种颜色对所有点染色,其中同一个联通块的点必须染成一种颜色,那么$f(S)$等于
这个图的染色方案数。同时,Alice喜欢奇数,所以当$S$的大小为奇数时,Alice的分值加上$f(S)$,否则Bob的分值
加上$f(S)$。求最后Alice的分值减去Bob的分值的值模$10^9+7$的结果。数据规模$n,m \le 10^6$。 \par

我们来看一些最直接的想法。一方面,若按照题目模拟,则需要枚举边集$E$的子集,
复杂度太高不能承受;另一方面,若枚举边数,很容易发现,即使两个边集中的
边数一样,它们所形成的联通块数不一定一样。我们发现这个问题很难入手,这时我们
应该写出问题的数学形式,看能否发现熟悉的东西。 \par

对题意稍加分析,我们可以发现,“同一个联通块的点必须染成一种颜色”等价于“一条边
连接的两个点必须染成一种颜色”,这一点非常重要。于是,我们可以设一些变量。设
$x_i$表示第$i$个点的颜色,那么一条边$(i,j)$就表示了一个条件$[x_i=x_j]$。令
$F(C)$表示条件$C$下,这个图的染色方案数,其中$C$表示一系列$[x_i=x_j]$的集合。
同时我们记Alice的得分为$scoreA$,Bob的得分为$scoreB$,那么我们可以直接按照题意
来写表达式,如屏幕所示,我们对边集$E$的所有子集求和,把染色方案用提到的$F$函数
代替。而我们的目标是求$scoreA-scoreB$,我们不妨将两式做差,容易发现奇偶性的条件
,因为正负号的原因,可以用$-1$的幂次来表示,然后得到了答案的表达式。我们发现,
整个式子的难点在于后面的$F$函数上。 \par

我们考虑$F$函数的含义,对一个边集的子集$S$,它代表满足$S$中的所有的边对应的条件
的染色方案数。我们再次细化条件,令$T_e$表示满足$e$这条边对应的条件时,所有染色
方案组成的集合。容易知道,$F(S)$就代表的是一些$T_e$的交集的大小。 \par

那么我们可以把所有的边从$1$到$Q$标号,枚举集合中的边数,重写答案的表达式,如
屏幕所示。我们发现上式的右边与容斥原理的形式相同,于是可以尝试逆向运用容斥原理,
从式子的右边推到左边,可以发现我们的答案其实是求一些集合的并集的大小,然后
我们再来考虑这个并集的含义,是我们在统计一个图$n$个点的染色方案数,使得至少一条
边对应的条件成立,即至少两个点的颜色相同。 \par

我再重新描述一下我们得到的新问题:$n$个点,$m$种颜色,至少两个点的颜色相同,图的染色方案数。我们通过一步简单的补集转化,先求两两颜色不同的的方案数,再用总的
方案数去减,这一步就非常简单了,如屏幕所示。 \par

回顾我们的解题过程。首先,我们发现直接入手比较困难,所以选择写出问题的数学形式,
用一些设变量或者换元的小技巧进行化简;然后,我们逆向使用了容斥原理,这种思路比较
新颖,较难想到;最后,我们发现了问题的本质,算出了答案。 \\

在游戏这道题目,我们以一种不同的方式运用了容斥原理。从这个问题中我发现容斥原理
背后蕴藏着一些思想:解决问题时,我们既可以用“隔离法”,拆分条件,解决子问题,再
合并答案;也可以用“整体法”,对所求的式子整体感知,逆向思考,合并已知条件,找出
问题的本质。这里,体现的都是转化的思想。 \par

另外,在论文中,我还探究了容斥原理在数论和概率论中的推广,这告诉我们不同领域间
存在或多或少的联系,在学习过程中,要多联想,多思考,体会所学算法背后蕴含的思想,
探寻其本质,形成自己的知识网络,这对我们是大有裨益的。 \\

最后,在这里我想感谢父母对我的养育,感谢我的教练成都七中的张君亮老师以及
其他所以给予我帮助的老师,感谢罗雨屏、李凌霄、钟皓曦等同学的帮助,感谢CCF
给我一个展示自己的机会,最后感谢在座各位的认真听讲。谢谢你们! \par

我的演讲结束,欢迎提问。

\end{document}
