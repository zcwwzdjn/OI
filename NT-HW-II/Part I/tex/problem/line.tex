\section{Line}
\subsection{题目描述}
{\itshape
Wayne喜欢排队……不对,是Wayne所在学校的校长喜欢看大家排队,尤其是在操场上站方阵。 \par
某日课间操时,校长童心大发想了一个极具观赏性的列队方案,如下: \par
\begin{enumerate}
\item 方阵排成$N$行,每行恰好$M$个学生。
\item 由于校长喜欢女孩子,所以在一行上不能有连续$P$个男生。
\item 由于校长喜欢女孩子,所以在校长看来,一列全是男生是不好的,全男生的列数不能超过$Q$。
\end{enumerate}
\par
Wayne因为感冒了所以不用参加列队,不过他看着大家排队排得不亦乐乎,于是他想知道,在男女生数目无限制的情况下,有多少种列队方案? \par
两种方案被视作不同,表明存在至少一个二元组$(i,j)$而两种方案中第$i$行第$j$列的同学性别不同。另外,因为答案可能很大,所以请把答案模$10^9 + 7$。
}
\subsection{输入格式}
{\itshape
输入仅一行$4$个正整数,依次是$N,M,P,Q$。
}
\subsection{输出格式}
{\itshape
输出仅一行,表示答案。
}
\subsection{样例输入}
{\tt
2 3 3 1
}
\subsection{样例输出}
{\tt
46
}
\subsection{数据规模}
{\itshape
对于$5\%$的数据,满足$P = 1$。 \par
对于另外$10\%$的数据,满足$N \times M \le 20$。 \par
对于另外$15\%$的数据,满足$N \le 2$,$M \le 10^6$。 \par
对于另外$10\%$的数据,满足$N \le 2$。 \par
对于另外$20\%$的数据,满足$N \le 4$,$P \le 2$,$Q \le 2$。 \par
对于$100\%$的数据,满足$1 \le N \le 8$,$1 \le M \le 10^{18}$,$1 \le P \le 3$,$0 \le Q \le 3$。
}
