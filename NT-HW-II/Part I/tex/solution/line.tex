\documentclass[a4paper]{article}
\usepackage{ctex}
%\usepackage[slantfont, boldfont, CJKtextspaces, CJKmathspaces]{xeCJK}
\usepackage{fontspec,xunicode,xltxtra}
\usepackage[pagestyles]{titlesec}
\usepackage{indentfirst}
\usepackage[top=1in,bottom=1in,left=1.25in,right=1.25in]{geometry}
\usepackage{amsmath}
%\usepackage[usenames,dvipsnames]{color}
\usepackage[colorlinks,linkcolor=red,anchorcolor=Blue,citecolor=green]{hyperref}

%\setCJKmainfont[BoldFont={黑体}, ItalicFont={楷体_GB2312}]{宋体}
\setmainfont{Liberation Serif} 
\setsansfont{Liberation Sans} 
\setmonofont{文泉驿等宽微米黑}

%\punctstyle{kaiming} % 开明式标点格式 
\renewcommand{\today}{}

\begin{document}

\title{Line解题报告}
\author{成都七中\ \  王迪}
\maketitle
\tableofcontents

\newpage

\section{题目大意}
统计满足如下要求的$N$行$M$列$01$矩阵数目:
\begin{itemize}
\item 每一行都不能有连续$P$个$0$;
\item 全是$0$的列数目不能超过$Q$。
\end{itemize}
\par
有$5\%$的数据,$P=1$; \par
有$10\%$的数据,$N \times M \le 20$; \par
有$15\%$的数据,$N \le 2$,$M \le 10^6$; \par
有$10\%$的数据,$N \le 2$; \par
有$20\%$的数据,$N \le 4$,$P \le 2$,$Q \le 2$; \par
对于$100\%$的数据,$N \le 8$,$M \le 10^{18}$,$P \le 3$,$Q \le 3$。 \par

\section{基础知识}
这篇题解假定大家了解矩阵、矩阵乘法,以及如何用矩阵乘法优化递推。请不了解的同学自行学习。

\section{算法分析}
这是一个组合计数问题,入手点很多,我们根据数据规模来进行分析。

\section{$5\%$的算法}
注意到有$5\%$的数据满足$P=1$,即不能有连续$1$个$0$,也就是说只能填$1$,所以答案就是$1$。 \par
时间复杂度$O(1)$,期望得分$5$分。

\section{$15\%$的算法}
有$10\%$的数据满足$N \times M \le 20$,这意味着我们可以枚举所有可能的矩阵进行判断。 \par
时间复杂度$O(2^{NM}NM)$,结合之前的算法可以得到$15$分。

\section{$30\%$的算法}
有$15\%$的数据满足$N \le 2$,$M \le 10^6$,我们可以考虑动态规划。为了方便就只讨论$N=1$的情况。 \par
记$dp[i][j][k]$表示已经填了前$i$列,第$i$列之前最近的一个$1$离$i$的距离为$j$,已经有$k$列全是$0$,的方案数。那么转移就考虑下一个位置填$0$还是$1$。 \par
对于$N=2$的情况则再加上一维记录第$2$行的情况。时间复杂度$O(P^NQM)$,结合之前的算法可以得到$30$分。

\section{$40\%$的算法}
有$10\%$的数据满足$N \le 2$,但是$M$可以达到$10^{18}$,我们可以考虑矩阵优化递推过程。 \par
由上一部分知动态规划,若按照已填列数划分阶段,则每一阶段的状态数是$O(P^NQ)$的,因为$N$比较小,所以可以手算出各种情况下的转移矩阵,再使用快速幂矩阵乘法加速递推。时间复杂度$O((P^NQ)^3\log{M})$,期望得分$40$分。

\section{$60\%$的算法}
有$20\%$的数据$N$已经达到了$4$,手算转移矩阵计算量过大,我们可以考虑用程序帮助我们找到转移矩阵。 \par
具体做法就是,我们枚举一个状态,再枚举下一列$N$个数的所有取值,算出它能转移到的状态。 \par
时间复杂度是$O((P^NQ)^3\log{M} + P^NQ \cdot 2^NN)$,期望得分$60$分。

\section{$100\%$的算法}
注意到对于极限数据,若仍用之前的算法,可以发现状态数过多,矩阵乘法无法承受。 \par
于是我们着手优化状态的表示。这里只讨论$P=2$的情况,其它的情况则是类似的。 \par
在表示状态时,我们用$N$维来表示了每一行离当前位置最近的$1$到当前位置的距离,我们称之为此时一行的$W$值,注意到这个值只有$0$和$1$两种取值。而进一步注意到,我们在处理转移时,对于不同的行,只要它们的$W$值相同,它们产生的转移是类似的!也就是说,真正对我们有用的,是$W=0$的行的数目和$W=1$的行的数目。
于是我们可以简化状态了:$(a,b,c)$三元组表示$W=0$的行有$a$个,$W=1$的行有$b$个,全是$0$的列已有$c$个。 \par
这样极大地压缩了状态。实际测试中极限数据中状态数仅有$180$。 \par
时间复杂度$O(S^3\log{M} + S \cdot 2^NN)$,其中$S$是实际的状态数,期望得分$100$分。

\end{document}
